\documentclass[11pt,a4paper,titlepage]{article}

\usepackage[utf8]{inputenc}
\usepackage[english]{babel}
\usepackage{csquotes}

\usepackage{listings}
\usepackage{xcolor}
\usepackage[none]{hyphenat}
\usepackage{textcomp}
\usepackage{parskip}

%Library
\usepackage[backend=bibtex,style=numeric]{biblatex}
\addbibresource{document.bib}

\setlength{\parskip}{1em}

\begin{document}

The main goal is that students use this progress report to schedule their work and allow supervisors an opportunity to comment on the approach taken by students and the level of progress made.

	Using an agile methodology will be an important part of the development of this project as it provides a far superior workflow as opposed to far more static methodologies such as the waterfall method.

Write about requirements

	For a game engine to be a complete engine a wide variety of features are required especially when a high level scripting language is the only input for developers to use when creating a game.

	Documentation is a requirement for people to be able to use the engine to create games.

Write about lib choices for each part

	Parts of the engine will be handled by external libraries, an important property of these libraries is performance and modernness as well as cross platform capability. Ideally the libraries will be open source.

	Rendering, physics, scripting and sound will be handled by external libraries and implemented in the game engine. For rendering and sound there are several choices: SDL, SFML, DirectX and OpenGL with a sound library. SDL is widely used and has very good performance but it is written in pure C so has no object oriented code and is therefore quite dated when compared to SFML. DirectX is only available on the windows and Xbox platforms and is therefore not ideal for this project as cross platform is a requirement. OpenGL with a sound library has the best performance of all of the options but is also the most difficult to use and both SDL and SFML wrap OpenGL and provide access to the context so would be better options.

	SFML or Simple and Fast Multimedia Library is written in C++ and is available on Linux, Mac, Windows and FreeBSD and is fully open source with its code available on github. SFML handles everything from window management and input to sound and networking, the performance is very similar to that of SDL but provides a modern object oriented API unlike SDL. These reasons make SFML the perfect choice for the projects requirements.

Write what needs to be done and in what order

	The first step to creating a game engine is to get each of the libraries working correctly in a single project.

	The engine will require a variety of features including an entity management system such as the entity component system, this will allow for dynamic game objects that are easily pieced together by a developer. This system will need to manage a wide variety of entities in various forms including drawing, animating, playing sounds and others.

	Lua script management will be entirely handled within the core of the game engine allowing for creating and removing objects, managing object behaviour, physics interactions, UI, the camera and much more. The engine will be a completely open source and will be expandable and improvable by the community as a result.

	For the UI an external library created to provide UI capabilities to SFML could be used such as SFGUI or TGUI. This will need binding for use in Lua which could prove to be challenging, however getting the minimum working example would not be too difficult.

	Loading and saving of the game state should be integrated into the game engine to speed up development time for the game developer. Level loading should also be integrated, an external open source level editor application could be used for the creation of the levels, a json or other data file could then be loaded into the game engine reducing the development time even further. Settings should also be managed by the game allowing the Lua developer to load and save any required key value pairs using a provided API, by implementing these features in the game engine the game developer can focus more on the actual game rather than semantics.

	Documentation should be completed after the completion of every Lua accessible feature to ensure that it is kept on top of. Some kind of automatic documentation tool could be used similar to that of Golang's GoDoc or Java's javadoc, this would allow for the the documentation to stay up to date given any changes to the underlying code.
		
	\printbibliography{}
\end{document}