\documentclass[11pt,a4paper,titlepage]{article}

\usepackage[utf8]{inputenc}
\usepackage[english]{babel}
\usepackage{csquotes}

\usepackage{listings}
\usepackage{xcolor}
\usepackage[none]{hyphenat}
\usepackage{textcomp}
\usepackage{parskip}

\author{Cyrus Hanlon}
\title{A Scriptable 2D Game Engine}

%Library
\usepackage[backend=bibtex,style=numeric]{biblatex}
\addbibresource{document.bib}

\setlength{\parskip}{1em}

\begin{document}
	\maketitle
	
	\definecolor{listinggray}{gray}{0.9}
	\definecolor{lbcolor}{rgb}{0.95,0.95,0.95}
	\lstdefinestyle{C++}{
		language=[GNU]C++,
		backgroundcolor=\color{lbcolor},
		captionpos=b,
		tabsize=4,    
		%   rulecolor=,
		basicstyle=\footnotesize,
		upquote=true,
		aboveskip={1.5\baselineskip},
		columns=fixed,
		showstringspaces=false,
		extendedchars=false,
		breaklines=true,
		numbersep=5pt,
		prebreak = \raisebox{0ex}[0ex][0ex]{\ensuremath{\hookleftarrow}},
		frame=lines,
		numbers=left,
		showtabs=false,
		showspaces=false,
		showstringspaces=false,
		identifierstyle=\ttfamily,
		keywordstyle=\color[rgb]{0,0,1},
		commentstyle=\color[rgb]{0,0.392,0},
		stringstyle=\color{red},
		numberstyle=\color[rgb]{0.205, 0.142, 0.73},
	}
	\lstdefinestyle{Lua}{
		language=[5.2]Lua,
		backgroundcolor=\color{lbcolor},
		captionpos=b,
		tabsize=4,    
		%   rulecolor=,
		basicstyle=\footnotesize,
		upquote=true,
		aboveskip={1.5\baselineskip},
		columns=fixed,
		showstringspaces=false,
		extendedchars=false,
		breaklines=true,
		numbersep=5pt,
		prebreak = \raisebox{0ex}[0ex][0ex]{\ensuremath{\hookleftarrow}},
		frame=lines,
		numbers=left,
		showtabs=false,
		showspaces=false,
		showstringspaces=false,
		identifierstyle=\ttfamily,
		keywordstyle=\color[rgb]{0,0,1},
		commentstyle=\color[rgb]{0,0.392,0},
		stringstyle=\color{red},
		numberstyle=\color[rgb]{0.205, 0.142, 0.73},
	}
		
	There are many game engines that already exist but nothing combines ease of use with raw speed.
	
	From Unreal Engine~\cite{UE} with its complete design 
	versus much lighter weight engines such as LÖVE~\cite{LOVE}, there is a very wide range of technologies and techniques at work. There is a definitive gap between large heavy engines and the small and lightweight that I intend to fill. By having many features written into the core engine but keeping a lightweight and easy to use API at the front, it will be possible to quickly learn how to easily prototype games, as well as, achieve excellent real world performance.

	A very important part of any game engine is the interface between developer and the inner workings. There are many scripting languages that can be chosen from.
	
	point out useful techniques and algorithms that will be employed in the solution. 
		
	How to bind Lua?
			
	%useful?
	%https://codeplea.com/game-scripting-languages
	%http://www.academia.edu/5492685/An_Analysis_of_Scripting_Languages_for_Research_in_Applied_Computing

	\newpage

	Main languages to use
	C++
	C\#
	C
	Java
	objective C
	HTML5
	
	Scripting languages to use https://github.com/r-lyeh/scriptorium
		Lua
		C\# %hash needs escaping
		Pawn
		Squirrel
		Io
		Javascript
		GameMonkey
		AngelScript
		Scheme
		Action script
		TinyScheme
		C jit
		Wren
		Lily
		Python
		Java
		PHP

	\newpage

	Lua was created for extending applications, there was an increase in demand for customization and no was language available that combined procedural features with powerful data description facilities.~\cite{Ierusalimschy02software} Lua came from the languages SOL and DEL~\cite{Ierusalimschy01theevolution} both of which were made by the company Tecgraf.
	
	talk about sol and del
	
	Lua is widely used in games and game engines thanks to its simple learning curve, ease of implementation using the provided minimalistic C API and its fast execution when compared to similar languages. It is also used in many non game settings such as in embedded systems using eLua~\cite{eLua} thanks to its low memory footprint.
	
	Lua is not interpreted directly but is instead compiled into bytecode first, this is then run on the Lua virtual machine. Because of this Lua is a fast language especially with the use of LuaJIT~\cite{LuaJIT}.

	talk about the Lua virtual machine is register based

	talk about intro to lua syntax

	Classes do not exist in Lua but object oriented programming is possible through the use of first class value functions in conjunction with tables. Multiple inheritance is possible through the use of metatables.

	Lua's support for coercion allows for the automatic conversion of strings and numbers, this helps developers with rapid development as they would not need to have to think about converting types in many situations when Lua itself handles it for them.

	A coroutine is similar to a thread in that it has its own line of execution, its own stack, its own local variables, and its own instruction pointer but shares global variables and almost anything else with other coroutines. The main difference to a thread is that a program with threads runs things concurrently while a coroutine in Lua is collaborative and at any given time only 1 is running~\cite{Ierusalimschy:2013:PLT:2502646}. While there may not be a performance gain from using coroutines, it allows a developer to easily orchestrate multitasking that would otherwise be difficult to implement.

	\newpage
	Lua has some syntactic sugar which helps to simplify what the Lua developer will have to interact with.

	\lstinputlisting[style=Lua, label={lst:SyntacticSugarSample1}, caption={Syntactic Sugar Sample 1}, firstline=1, lastline=8]{codeSamples/1.lua}
	In listing~\ref{lst:SyntacticSugarSample1} we can see how to inline declare a table with a sub table that is also inline declared.

	\lstinputlisting[style=Lua, label={lst:SyntacticSugarSample2}, caption={Syntactic Sugar Sample 2}, firstline=17, lastline=21]{codeSamples/1.lua}
	We can see how to declare a member function using some syntactic sugar in listing~\ref{lst:SyntacticSugarSample2}. Which is equivalent to listing~\ref{lst:SyntacticSugarSample3}.

	\lstinputlisting[style=Lua, label={lst:SyntacticSugarSample3}, caption={Syntactic Sugar Sample 3}, firstline=10, lastline=15]{codeSamples/1.lua}

	There are also several ways to access entries within a table.

	\lstinputlisting[style=Lua, label={lst:SyntacticSugarSample4}, caption={Syntactic Sugar Sample 4}, firstline=23, lastline=25]{codeSamples/1.lua}

	\newpage

	talk about:

	Advanced features:
	Cross platform due to its interpreter being written in ANSI C
	Functions are first class values
	Garbage collection
	Closures
	Proper tail calls to allow for recursion
	Dynamic Module Loading

	\newpage
	
		Performance of
			Love2D
			Unreal 
			Engine 
			Unity 
			Cry Engine 
			Source (Garrysmod)
			Game Studio
			id Tech (5/6?)
			MonoGame
			Cocos
			OGRE
			
\newpage
\printbibliography{}
\end{document}