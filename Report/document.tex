\documentclass[11pt,a4paper,titlepage]{article}

\usepackage[utf8]{inputenc}
\usepackage[english]{babel}
\usepackage{csquotes}

\usepackage{listings}
\usepackage{xcolor}
\usepackage[none]{hyphenat}
\usepackage{textcomp}
\usepackage{parskip}

\author{Cyrus Hanlon}
\title{A Scriptable 2D Game Engine}

%Library
\usepackage[backend=bibtex,style=numeric]{biblatex}
\addbibresource{document.bib}

\setlength{\parskip}{1em}

\begin{document}
	\maketitle
	
	\definecolor{listinggray}{gray}{0.9}
	\definecolor{lbcolor}{rgb}{0.95,0.95,0.95}
	\lstset{
		backgroundcolor=\color{lbcolor},
		captionpos=b,
		tabsize=4,    
		%   rulecolor=,
		language=[GNU]C++,
		basicstyle=\footnotesize,
		upquote=true,
		aboveskip={1.5\baselineskip},
		columns=fixed,
		showstringspaces=false,
		extendedchars=false,
		breaklines=true,
		numbersep=5pt,
		prebreak = \raisebox{0ex}[0ex][0ex]{\ensuremath{\hookleftarrow}},
		frame=lines,
		numbers=left,
		showtabs=false,
		showspaces=false,
		showstringspaces=false,
		identifierstyle=\ttfamily,
		keywordstyle=\color[rgb]{0,0,1},
		commentstyle=\color[rgb]{0,0.392,0},
		stringstyle=\color{red},
		numberstyle=\color[rgb]{0.205, 0.142, 0.73},
	}

	\begin{lstlisting}
	#include<stdio.h>
	#include<iostream>
	// A comment
	int main(void)
	{
		printf("Hello World\n");
		return 0;
	}
	\end{lstlisting}
	
	The report should describe (concisely) existing work with the aim of 
	
	(1) contextualising the project.
	
		Where does it fit in the bigger picture, why does it exist?

		There are many game engines that already exist but nothing combines ease of use with raw speed.
		
		From Unreal Engine~\cite{UE} with its complete design 
		versus much lighter weight engines such as LÖVE~\cite{LOVE}, there is a very wide range of technologies and techniques at work. There is a definitive gap between large heavy engines and the small and lightweight that I intend to fill. By having many features written into the core engine but keeping a lightweight and easy to use API at the front, it will be possible to quickly learn how to easily prototype games, as well as, achieve excellent real world performance.
	
	(2) point out useful techniques and algorithms that will be employed in the solution. 
		
		How to bind Lua?
		
	(3) analyse critically and compare existing work (e.g., conference and journal articles, pieces of software) that address a (significantly) similar	problem to the one addressed by one’s project.
	
		articles/papers on scripting languages
		pros and cons of each language
			eco system and ease of learning
			ease of implementation
		games that already use languages
		
		A very important part of any game engine is the interface between developer and the inner workings. There are many scripting languages that can be chosen from.
		
		%https://codeplea.com/game-scripting-languages
		%http://www.academia.edu/5492685/An_Analysis_of_Scripting_Languages_for_Research_in_Applied_Computing
		\newpage
		Discuss Lua as a language and why its a good choice to use for this project

		Lua was created as a language for extending applications, there was an increase in demand for customizable applications and no language available that combined procedural features with powerful data description facilities.~\cite{Ierusalimschy02software} It is lightweight due to its low memory footprint and is easy to implement using its C API.

		Functions are first class values
		Cross platform due to its interpreter being written in ANSI C

		Advanced features:
		Garbage collection
		Closures
		Proper tail calls to allow for recursion
		Coercion
		Coroutines (not multiple threads)
		Dynamic Module Loading

		Classes do not exist in Lua but object oriented programming is possible through the use of first class value functions in conjunction with tables. Multiple inheritance is possible through the use of metatables.
		
		Lua came from the languages SOL and DEL~\cite{Ierusalimschy01theevolution} both of which were made by Tecgraf.

		Lua has some syntactic sugar which helps to simplify what the Lua developer will have to interact with
		table[\"place\"] is identical to table.place
		table {
			\"foo\" = 45,
			\"tab\" = {
				x = 23
			}
		}
		table.tab.x is 23

		table.func(self, 100)
		table:func(100)
		They are the same

		Lua is not interpreted directly but is instead compiled into bytecode first. This is then run on the Lua virtual machine, due to the fact that it is first compiled into bytecode, Lua is a fast language.

		The Lua virtual machine is register based

		The Lua C API is very minimalistic and is intended for embedding Lua into applications.


		
		Main languages to use
			C++
			C\#
			C
			Java
			objective C
			HTML5
			

		Scripting languages to use https://github.com/r-lyeh/scriptorium
			Lua
			C\# %hash needs escaping
			Pawn
			Squirrel
			Io
			Javascript
			GameMonkey
			AngelScript
			Scheme
			Action script

			TinyScheme
			C jit
			Wren
			Lily
			Python
			Java
			PHP
	
		Performance of
			Love2D
			Unreal 
			Engine 
			Unity 
			Cry Engine 
			Source (Garrysmod)
			Game Studio
			id Tech (5/6?)
			MonoGame
			Cocos
			OGRE
			
\newpage
\printbibliography{}
\end{document}