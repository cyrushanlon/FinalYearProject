\documentclass[11pt,a4paper,titlepage]{article}

\usepackage[utf8]{inputenc}
\usepackage[english]{babel}
\usepackage{csquotes}

\usepackage{listings}
\usepackage{xcolor}
\usepackage[none]{hyphenat}
\usepackage{textcomp}
\usepackage{parskip}

\author{Cyrus Hanlon}
\title{A Scriptable 2D Game Engine}

%Library
\usepackage[backend=bibtex,style=numeric]{biblatex}
\addbibresource{document.bib}

\setlength{\parskip}{1em}

\begin{document}
	\maketitle
	
	\definecolor{listinggray}{gray}{0.9}
	\definecolor{lbcolor}{rgb}{0.95,0.95,0.95}
	\lstdefinestyle{C++}{
		language=[GNU]C++,
		backgroundcolor=\color{lbcolor},
		captionpos=b,
		tabsize=4,    
		%   rulecolor=,
		basicstyle=\footnotesize,
		upquote=true,
		aboveskip={1.5\baselineskip},
		columns=fixed,
		showstringspaces=false,
		extendedchars=false,
		breaklines=true,
		numbersep=5pt,
		prebreak = \raisebox{0ex}[0ex][0ex]{\ensuremath{\hookleftarrow}},
		frame=lines,
		numbers=left,
		showtabs=false,
		showspaces=false,
		showstringspaces=false,
		identifierstyle=\ttfamily,
		keywordstyle=\color[rgb]{0,0,1},
		commentstyle=\color[rgb]{0,0.392,0},
		stringstyle=\color{red},
		numberstyle=\color[rgb]{0.205, 0.142, 0.73},
	}
	\lstdefinestyle{Lua}{
		language=[5.2]Lua,
		backgroundcolor=\color{lbcolor},
		captionpos=b,
		tabsize=4,    
		%   rulecolor=,
		basicstyle=\footnotesize,
		upquote=true,
		aboveskip={1.5\baselineskip},
		columns=fixed,
		showstringspaces=false,
		extendedchars=false,
		breaklines=true,
		numbersep=5pt,
		prebreak = \raisebox{0ex}[0ex][0ex]{\ensuremath{\hookleftarrow}},
		frame=lines,
		numbers=left,
		showtabs=false,
		showspaces=false,
		showstringspaces=false,
		identifierstyle=\ttfamily,
		keywordstyle=\color[rgb]{0,0,1},
		commentstyle=\color[rgb]{0,0.392,0},
		stringstyle=\color{red},
		numberstyle=\color[rgb]{0.205, 0.142, 0.73},
	}

	\tableofcontents{}
	\newpage

	\section{Introduction}
			
		There are many game engines that already exist but nothing combines ease of use with raw speed.
		
		From Unreal Engine~\cite{UE} with its complete design 
		versus much lighter weight engines such as LÖVE~\cite{LOVE}, there is a very wide range of technologies and techniques at work. There is a definitive gap between large heavy engines and the small and lightweight that I intend to fill. By having many features written into the core engine but keeping a lightweight and easy to use API at the front, it will be possible to quickly learn how to prototype games, as well as, achieve excellent real world performance.

		A very important part of any game engine is the interface between developer and the inner workings. There are many scripting languages that can be chosen from and there are many important factors that come into play when selecting a language to use, the most important factors include: popularity, speed and learning curve. Lua is at the top of the pile, especially when it comes to games. Other languages such as JavaScript and Python are extremely popular in other areas but are not anywhere near as often used in games as they are more difficult to bind.

		Lua
		Pawn
		Squirrel
		Io
		Javascript
		GameMonkey
		AngelScript
		Scheme
		TinyScheme
		Wren
		Lily
		Python
		
		point out useful techniques and algorithms that will be employed in the solution. 
			
		How to bind Lua?
		What needs to be accessed through the scripting language?
				
		%useful?
		%https://codeplea.com/game-scripting-languages
		%http://www.academia.edu/5492685/An_Analysis_of_Scripting_Languages_for_Research_in_Applied_Computing
		
	\newpage

	\section{Lua}
		\subsection{About Lua}
			Lua was created for extending applications, there was an increase in demand for customisation and no was language available that combined procedural features with powerful data description facilities.~\cite{Ierusalimschy02software} Lua came from the languages SOL and DEL~\cite{Ierusalimschy01theevolution} both of which were made by the company Tecgraf.
			
			SOL is an acronym for Simple Object Language was a specialized data description language used for customising software. The SOL syntax was strongly influenced by the syntax of BiBTeX~\cite{Lamport:1989:LDP:63364}. DEL is an acronym for Data Entry Language, an entity in DEL is essentially an object or record in other languages~\cite{Ierusalimschy01theevolution}. Lua was created as a unique combination of the two more primitive languages, it was designed to avoid unique syntax so that anyone could use it effectively with limited training and for the language to have built in type checking.
			
			Lua is written in C and so, is very cross platform~\cite{LuaSource}, it is used in projects ranging from Cheat Engine~\cite{CheatEngine}, an open source memory scanner and hex editor for Windows, to Android and iOS games using something such as Cocos2d~\cite{COCOS2D}, an open source software framework used in the creation of apps or games. Therefore Lua is widely used thanks to its simple learning curve, ease of implementation using the provided minimalistic C API and its fast execution when compared to similar languages. It is also used in many non game settings such as in embedded systems using eLua~\cite{eLua} thanks to its low memory footprint.
				
		\subsubsection{Running Lua}
			Lua is not interpreted directly but is instead compiled into bytecode, this is then run on the Lua virtual machine. Because of this Lua is a fast language, considering how high level it is. Lua used a stack based virtual machine initially but with the release of Lua 5.0, Lua began to use a register based machine. It also uses a stack for allocating activation records where the registers live~\cite{IerusalimschyImplementation}.

			Register based vs stack based

			Lua can also be used from the command line which allows for productivity scripts to be written and used without the need for it to be embedded into another application. Lua can also be used as a standalone program using the tool srLua which was created by one of the Lua authors~\cite{tecgrafLibAndTools}, this allows the language to be used almost anywhere on almost anything.

		\subsubsection{Syntax}
			The syntax of Lua was designed so that it combined simple procedural syntax with powerful data description constructs, the general syntax is very easy to learn with its roots in SOL and DEL, both were designed for ease of use. It borrows much of its syntax from other languages, its comment syntax, for example, is similar to that of Haskell and SQL. It's control flow is similar to that of pascal using keywords such as \texttt{then} and \texttt{end} for \texttt{if} statements. It is, however, renowned for being very easy to write with its simple syntax and extremely powerful table constructs. A quirk when working with Lua is the fact that tables start at an index of 1 instead of 0 as is very common throughout programming languages.

			An example of Lua's control statements are shown in listing~\ref{lst:SyntaxSample1}. The use of \texttt{elseif} as a single keyword is useful and avoids the need to use an \texttt{end} per \texttt{if}. The conditions in each of the statements do not need to be wrapped in parenthesis and allow for logical operators. 

			\lstinputlisting[style=Lua, label={lst:SyntaxSample1}, caption={Lua Syntax Sample 1}, firstline=63, lastline=80]{codeSamples/1.lua}

			There are two types of \texttt{for} loop in Lua, one is a numeric \texttt{for} and is shown in listing~\ref{lst:SyntaxSample1}, the other is the generic \texttt{for} that uses an iterator and allows for the traversal of all values. The numeric \texttt{for} uses a very simple syntax and so is very easy to write, the first argument is the starting value of the iteration variable, the second is the end value and the third is the value to add with each loop.

			The logical operators in Lua are \texttt{and}, \texttt{or} and \texttt{not}. Logical operators and control statement expressions take \texttt{nil} and \texttt{false} as \texttt{false} and anything else as \texttt{true}~\cite{LuaMan} This allows for some unique syntax that more advanced Lua programmers will use to reduce the amount of code required.

			Each of the three logical operators behave differently when used in this way. The \texttt{and} operator returns its first argument if it is \texttt{false} and otherwise returns its second argument. The \texttt{or} operator returns its first argument if this value isn't \texttt{false} and returns its second argument otherwise. The \texttt{not} operator always returns \texttt{true} or \texttt{false} based on the argument. The Lua 5.3 manual~\cite{LuaMan} has an excellent set of examples as shown in listing~\ref{lst:SyntaxSample2}.

			\lstinputlisting[style=Lua, label={lst:SyntaxSample2}, caption={Lua Syntax Sample 2 (taken from \cite{LuaMan})}, firstline=82, lastline=89]{codeSamples/1.lua}

			Using this behaviour of logical operators allows for quickly and easily assigning variables using, for example, a configuration file.

			multiple assignment local a, b = 5, 6

			bitwise operators

			.. concatenation

			iterators

			Lua has some syntactic sugar which helps to simplify what the Lua developer will have to interact with.
			
			\lstinputlisting[style=Lua, label={lst:SyntacticSugarSample1}, caption={Syntactic Sugar Sample 1}, firstline=1, lastline=8]{codeSamples/1.lua}
			In listing~\ref{lst:SyntacticSugarSample1} we can see how to inline declare a table with a sub table that is also inline declared.

			\lstinputlisting[style=Lua, label={lst:SyntacticSugarSample2}, caption={Syntactic Sugar Sample 2}, firstline=17, lastline=21]{codeSamples/1.lua}
			We can see how to declare a member function using some syntactic sugar in listing~\ref{lst:SyntacticSugarSample2}. Which is equivalent to listing~\ref{lst:SyntacticSugarSample3}.

			\lstinputlisting[style=Lua, label={lst:SyntacticSugarSample3}, caption={Syntactic Sugar Sample 3}, firstline=10, lastline=15]{codeSamples/1.lua}

			There are also several ways to access entries within a table all of which can be mixed and matched based on the programmers preference.

			\lstinputlisting[style=Lua, label={lst:SyntacticSugarSample4}, caption={Syntactic Sugar Sample 4}, firstline=23, lastline=25]{codeSamples/1.lua}

			the \# operator on tables

		\subsubsection{Functions}
			Lua allows for a function to return multiple values, this is quite a strange concept and can have some downsides, it can make it quite difficult to correctly name a function to reflect the return values. It is a very powerful feature and if handled correctly can improve productivity especially when combined with Lua's variable number of arguments. An example is shown in listing~\ref{lst:Functions1}.
			\lstinputlisting[style=Lua, label={lst:Functions1}, caption={Functions Sample 1}, firstline=97, lastline=105]{codeSamples/1.lua}

			Functions, in Lua, are first class values with proper lexical scoping~\cite{Ierusalimschy:2013:PLT:2502646}. A first class value in Lua means that a function is essentially the same as any other regular variable, this allows a Lua developer to pass functions into other functions as arguments, return functions, use functions of keys and values of tables and can be assigned to a variable. This is a very valuable trait for a high level scripting language to have as it allows for very interesting techniques such as functional programming.

			Lua has proper tail calls which allows for recursion. If the last thing that a function does is call another function, the calling function is no longer required and is no longer required on the stack, this allows for an infinite recursion without overflowing the stack~\cite{Ierusalimschy:2013:PLT:2502646}. An example of a Lua tail call is shown in listing~\ref{lst:Functions2}.
			\lstinputlisting[style=Lua, label={lst:Functions2}, caption={Functions Sample 2}, firstline=91, lastline=95]{codeSamples/1.lua}

			Without proper tail calls, recursion could overflow the stack but with proper tail calls this is impossible and fits in perfectly with a high level scripting language such as Lua.

		\subsubsection{Tables}
			Tables are the only data structure in Lua but are extremely flexible and can be used to provide the functionality of almost any data structure as required. They are very similar in use to a hash map and allow for any variable of any type to be assigned to a key or value within the same table with the exception of \texttt{nil}, it is also possible to treat a table similar to an array as shown in listing~\ref{lst:TablesSample1}.
			\lstinputlisting[style=Lua, label={lst:TablesSample1}, caption={Tables in Lua Sample 1}, firstline=27, lastline=36]{codeSamples/1.lua}

			Lua has a variety of built in functions for the use of manipulating and getting data from tables all available through the table library~\cite{LuaMan}.

			To loop over all key-value pairs in a table lua provides the \texttt{ipairs} and \texttt{pairs} iterators. Used with a \texttt{for} loop a Lua developer is provided with a powerful method of quickly and easily accessing anything within a table as shown in listing~\ref{lst:TablesSample2}.
			\lstinputlisting[style=Lua, label={lst:TablesSample2}, caption={Tables in Lua Sample 2}, firstline=38, lastline=61]{codeSamples/1.lua}

			The \texttt{pairs} function iterates over all elements in a table while the \texttt{ipairs} function will return all key-value pairs where the keys are numbers, starting at the first index and stopping when it hits the first nil value.~\cite{Ierusalimschy:2013:PLT:2502646} 
			
			Getting the length of a Lua array is found by calling \texttt{table.getn(tab)} however, to find the size of a table with mixed key types or non-consecutive indices, a generic \texttt{for} loop with a count variable would have to be used. In cases where mixed key types are used it shouldn't be required to know the size as the pairs cannot be accessed by a normal for loop in any case.

			Weak tables are tables that consist of elements that are weak references, this can be the keys or the values, if the only reference left to an object is a weak reference the garbage collector collects it. They are often used when a word needs to be reserved without actually creating an object in memory.
			
		\subsubsection{Coercion}
			Lua's support for coercion allows for the automatic conversion of strings and numbers, this helps developers with rapid development as they would not need to have to think about converting types in many situations when Lua itself handles it for them.

			examples

		\subsubsection{Metatables and Object Oriented Programming}
			All tables in Lua are permitted to have a metatable\cite{Ierusalimschy:2013:PLT:2502646}. They allow for changing the behaviour of various aspects of a lua table, such as the add operator, by setting the \texttt{\_\_add} value to a new function this code will be called instead of the default add functionality as is shown in listing~\ref{lst:MetaTables1}.
			\lstinputlisting[style=Lua, label={lst:MetaTables1}, caption={Metatables Sample 1}, firstline=107, lastline=117]{codeSamples/1.lua}

			These functions are called metamethods and allow for any interaction with a table to be altered. While classes do not exist in Lua, using a meta table and first class value functions, behaviour similar to object oriented programming can be achieved especially with the use of factory methods. An example of a class is shown in listing~\ref{lst:MetaTables2}. On line 13 a table with the metatable \texttt{Dog} is created and returned to \texttt{myDog}.
			\lstinputlisting[style=Lua, label={lst:MetaTables2}, caption={Metatables Sample 2}, firstline=119, lastline=132]{codeSamples/1.lua}

			While this concept is very useful, if the \texttt{new} function is ever changed by mistake, a difficult to locate bug could be created.

		\subsubsection{Couroutines}
			A coroutine is similar to a thread in that it has its own line of execution, its own stack, its own local variables, and its own instruction pointer but shares global variables and almost anything else with other coroutines. The main difference to a thread is that a program with threads runs things concurrently while a coroutine in Lua is collaborative and at any given time only 1 is running~\cite{Ierusalimschy:2013:PLT:2502646}. While there may not be a performance gain from using coroutines, it allows a developer to easily orchestrate multitasking that would otherwise be difficult to implement.

			examples 

		\subsubsection{Dynamic Module and Library Loading}

			Intro 

			Lua allows for loading modules at run time that were not required at build time, this allows for extremely flexible program extensions. A module is made available through the use of an ordinary table with functions or variables being accessed through the table.

			Loading, compiling then executing Lua files from within Lua is easily possible and allows for things such as hot reloading code vastly improving the development rate as developers could just save a file and the Lua program would simply start using it.

			example of a simple module being created

		\subsubsection{Garbage Collection}

			Lua automatically manages memory, Lua destroys objects when all references to them are lost. There is no function to delete them manually, a program can only create objects which means that the developer will not need to bother with the majority of memory management~\cite{Ierusalimschy:2013:PLT:2502646}. Lua has no issues dealing with cyclic references unlike simple garbage collectors. Lua allows for manually calling the garbage collector using the \texttt{collectgarbage} function as even the best garbage collector isn't perfect, the Lua garbage collector sometimes needs some help. It is even possible to manually pause, stop and restart the garbage collector to allow for fine tuned adjustments to the use of memory in a Lua program, it is also possible to get the total amount of memory being used.

			examples of why you would manually do gc stuff

		\subsubsection{The Lua C API}

	\newpage

	\section{Game Engines}
		
			Performance of
				Love2D
				Unreal 
				Engine 
				Unity 
				Cry Engine 
				Source (Garrysmod)
				Game Studio
				id Tech (5/6?)
				MonoGame
				Cocos
				OGRE	
	\newpage

	\printbibliography{}
\end{document}